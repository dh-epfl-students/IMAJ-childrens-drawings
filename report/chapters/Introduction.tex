\section{Visual Arts}

Art stimulates thoughts, emotions, beliefs, or ideas through the senses in an individual. Various forms of art deal with different sensibilities of the human body. For example, paintings, sculptures, and photographs concern the visual organs. Music is consumed through auditory organs. Then there are art forms such as theatre, dance, drama, and movies that involve both senses. Since the dawn of humankind, art forms have evolved in diverse ways. Yet from cave paintings to today’s metaverse, the visual aspect of the art remains constant. 

While creating arts constitutes a critical part of the human experience and portrays the world around us in the present moment, studying them throws light on our culture, lives, and the past experiences of fellow humans. Besides, the study can inspire, reflect and serve us in designing the future. A wide range of activities, from fine arts to product designs, constitute visual arts; this work refers to paintings and drawings as visual arts. Studying the relations between artworks and comparing them across time and space has been a fundamental activity in Art History. Besides, many studies have already established the importance of visual arts in developing a creative mindset at a young age \cite{kamp_2015}, understanding the learning outcomes in children \cite{tyler_2012}, and shaping humans' personalities \cite{kindler_2003}.


\section{Digitization and Analysis}

Digitization and the online availability of art collections make the physical study of the art pieces a thing of the past. The markedly improved photographic and scanning technologies and their availability at comparatively inexpensive costs made digitization take a front seat at many libraries, galleries, museums, and archives. These digitized collections, especially publicly available ones like the WikiArt \cite{wikiart}, MET collection \cite{met_coll}, or the Rijks Museum collection \cite{rijks_coll}, have enabled all kinds of people to explore these art databases. 

Viewing, remembering, and identifying thousands of images, if not millions, is an impractical task for humans. In parallel to the hardware advancement (photographic devices), progress in software and algorithms related to machine vision helped to create tools to view, analyze and improve the understanding of visual arts. Among the computer vision algorithms, algorithms that use neural networks have achieved and beaten human performance on many tasks such as classification and recognition in the last decade. A subset of algorithms that use multiple layers of neural networks, dubbed Deep Learning, which requires large volumes of data, takes credit for achieving unimaginable improvements. Among the deep learning techniques, \emph{Convolutional Neural Networks} (CNN) pushed the limits of machines in solving vision-related problems. The ability of machine learning algorithms and the digital availability of vast art collections - due to digitization - opened a new avenue in the study of art.

The conjuncture of Art History and CNNs from deep learning became an active research area in the recent past. Multiple improved architectures of CNNs gave the state of the art performance in predicting the attributes of paintings \cite{saleh_2016, tan_2016} and specifically on style \cite{karayev_2014}, genre \cite{cetinic_2018}, and artist \cite{van_2015}. Shen et al., \cite{Shen2019DiscoveringVP} developed systems for object and near-duplicate detection of artworks. In the latest round of developments, new art was produced \cite{Elgammal2017CANCA, Tan2019ImprovedAF} using the art made by humans in the Generative Adversarial Networks \cite{Ian2014_GAN}. The examination of similarity between artworks and artists has received profound interest in the last five years. Seguin et al., \cite{seguin_2016} proposed a supervised deep learning technique to search for similar visual patterns through metric learning. Further, methods to search for visual patterns in an unsupervised setting were proposed \cite{gultepe_2018, Castellano2021VisualLR} and all these works can broadly fit into the problem of clustering. Although there are comparatively fewer studies, other research areas in this conjuncture are fake art detection \cite{elgammal_2018}, art to photo translation (and vice-versa) \cite{Tomei2019Art2RealUT, matteo_2019}, emotion detection in paintings \cite{HE_wei_2018} and generating descriptions of artworks \cite{noa_2019}. Santos et al. \cite{Santos2021ArtificialNN} provides a review of deep learning applications in visual arts.

\section{Children and Visual Arts}

Visual subjects like the people, animals, buildings, or surroundings serve as entry into this world for a child. Later, in school and private life, children view and produce various arts, which provides a rich learning domain for a young child \cite{angela_2008}. Often, many might not regard the young children's drawings as novel creations and even compare them with the works of great artists or artworks. It is almost impossible to pinpoint the factors that influenced the child to make a particular drawing. There will be spontaneous instinct, vivid imagination, the influence of daily life, friends, family, teachers, and other unknown elements. While it is difficult to quantify many of these personal factors, it is possible to find the influence of artists and art forms in the drawings. Art viewing and the abundance of information available about renowned artworks and artists could explain the impact of famous works in their drawings. At the same time, art schools use the famous works in teaching children and practice recreating them. These art classes could also be another reason that explains the resemblance between the drawings and famous artworks.

\section{Artistic Analysis of Children's Drawings}\label{chap:1:sec:children-art-analysis}

Humans perceive objects through patterns, and as mentioned earlier, art historians try and find morphological, stylistic, and semantic similarities in the artworks. Along with conveying meaning, these patterns can provide insights into artists and the influences they might have undergone in producing the work. Again, in the context of children, the drawings (even scribbles by babies) were used in psychological \cite{toku_2001}, pediatric, and education studies \cite{yang2006developmental}. In addition to the medical insights, investigating children's drawings helps to learn the impressions of historical, cultural, geopolitical, and socio-economic situations on the young children alongside exploring the themes, objects, artistic styles, and time period influences in their drawings. The unavailability of a large number of drawings produced by children makes such a study on a vast scale strenuous. This work attempts to move research in that direction.

Children move from scribbling to schematic representation and drawing realism to crises of adolescence stages during the growth and development of art in them \cite{lowenfeld1957creative} (as cited in \cite{kantner_gregory_2002}). In this process, the youngsters with the influence of popular cultural objects such as movies, music, and paintings try to recreate them\footnote{This work refers to the collections of popular paintings, posters of music albums, movies, dances, photographs of buildings, landmarks, people, and other famous cultural subjects collectively as popular/famous artworks.}. Analyzing and mining such references provides an understanding of cross-cultural influences on children and the propagation of art beyond borders. Finding the connections in a drawing also helps move from simply looking at it as a sketch to taking a deeper look into it and unraveling the hidden connotations. To this end, this project endeavors to create a system to identify the popular artwork references in the children's drawings.

\section{UNESCO Center in Troyes}

The \textit{Institut Mondial d'Art de la Jeunesse - Centre pour l'UNESCO Louis François} (World Youth Art Institute - Louis François Center for UNESCO) in the French city of Troyes aims to "promote creativity, develop artistic practices, and enhance the diversity of cultural expressions among young people." Located 140 kilometers from Paris and in the Champagne wine region, originally started as Cercle UNESCO de Troyes in 1978, \textit{Louis François Center for UNESCO} is the only UNESCO center in France. It has undergone several name changes until its current name in 2019 and became a UNESCO center in 1994. With an objective to \textit{inscribe childhood and youth in the Memory of Humanity}, the center provides educational resources and organizes multiple workshops and competitions \cite{centre_unesco_2020}. Their \textit{Concours international d’arts plastiques} (International visual art competition) is the largest and most popular activity among the activities organized by the center to promote its objectives.

\section{\foreignlanguage{french}{Graines d’artistes du monde entier}}

The international visual arts competition, titled \textit{Graines d’artistes du monde entier} (Seeds of artists from around the world), started locally in 1985 and became a national competition in 1992, and within two years, it evolved into an annual international competition. Since 1994, the center has accumulated the artistic expressions of young people under 25 years old, under varying themes each year, and rewards medals and diplomas to a hundred laureates.

As part of the World Art Institute of Youth's mission, they conserve the works of all young artists who have participated in the competition. All the submissions received as part of the competition, starting from 1994, are now part of their \textit{Mémoires du Futur} (Memories of the Future) art library. \textit{Mémoires du Futur} museum already contains more than 100,000 artistic productions by young people aged 3 to 25. The drawings are spread over a quarter-century and originated from 150 countries, making the collection spatially and temporally diverse. The UNESCO center started digitizing this unique and invaluable collection to take them to a broad audience \cite{centre_unesco_2020}. 

Making the drawings available on a digital platform accessible to everyone provides recognition to the children. On the other hand, it gives a chance to revisit the research on psychological, sociological, artistic, and historical aspects to improve the understanding of the creative notion of the child and its evolution on a broader scale with a large set of drawings. In addition, the digital copies provide an opportunity to study them using computer vision techniques. To this day, they have digitized more than 80\% of the collection they have acquired in the past 28 years. Chapter \ref{chap:3:AboutDrawings} provides a detailed description of the collection.

\section{Motivation}

Discovering the artwork references, even in small numbers, with high confidence is essential in providing a cultural context for the drawings. Experts can quickly point out the popular connections among a handful of drawing pieces. The task becomes tedious and impractical in the case of comparing the works of different aged artists rooted across the world with the famous cultural objects across countries.

Children's drawings greatly differ from paintings or standard image datasets containing photographs. Children use diverse styles, materials, and techniques and do not necessarily use the same method to recreate famous paintings. In addition, the detailing and morphological closeness of the drawing and artwork differ among children of different ages, making the comparison between them difficult. Learning specific patterns in images could lead to an unsuccessful comparison as the similarity between the drawing and the famous works are about correspondences which are neither among the low-level features such as color, texture, nor exact shapes and objects. At the same time, due to the minuscule number of examples available, the system can learn those pairings by heart. The comparision also differs from the sketch retrieval systems that commonly use pencil sketches for training and evaluation \cite{Yi2018SBIR} lacking the diversity in the different genres of drawing. This uniqueness characterizes the task at hand not as a standard image retrieval problem or seldom a typical machine learning task and places it in the territory of cross-domain drawing retrieval. Only a limited amount of work is available on such a subject (discussed in Section \ref{chap:2:related_work}), and the existing ones require heavy computation.

In light of these scenarios, cross-domain drawing retrieval is a clearly defined complex problem. However, the examples of drawing - artwork 
pairs show that they cannot fit into a singularly defined concept but needs a fluid boundary. In addition, the interpretations of similarity differ with an individual and the context, making it a necessary step to visit the primary sources and codify the constraints. These are typical traits of a Digital Humanities problem. The artwork matching is considered in the Digital Humanities domain and attempts to solve such an intricate problem using the digital computational tools, in this case, deep computer vision tools. 

\section{Thesis Goals}

This thesis 
\begin{itemize}
    \item primarily aims to devise techniques to identify famous artworks that are visually similar to the children's drawings and
    \item provide the first insights into the recently digitized database of children's drawings
\end{itemize}


The recent boom in digitization pushed researchers to create efficient pipelines to digitize documents containing text, art, images, and many other types of information. Then the focus shifts to providing functional access to digitized data, and this thesis form part of such ongoing efforts. However, it attempts to extract similar images in two different datasets different from the access based on traditional indexing using the metadata. Matching images that vary in texture, colors, method, and techniques pose a challenge to current visual image search methods. It is more pressing when one set of images are creations of children. The earlier works are also limited in dataset size and identifying the exact paintings children have recreated. The recently digitized massive collection of drawings will provide a chance to look at the problem coming over previous constraints and a case to furnish the first impressions about the dataset.


\section{Organization of the thesis}

The thesis constitutes seven chapters. The next chapter documents the related works, and the third chapter introduces and discusses more than 80,000 digitized children's drawings from the \textit{Louis François center for UNESCO}. The fourth chapter presents formal modeling of the similarity detection problem between drawings and paintings and the methods used in this study. The fifth and penultimate chapter describes the experiments and provides the results. The sixth chapter discusses the results, analyzes the limits, provides suggestions for the improvement of the methods used to identify the similarity, and examines the possible potential future work. Lastly, the seventh chapter summarizes the thesis while highlighting the contributions of this work. 